\documentclass{article}

\usepackage{amsmath}
\usepackage[margin=0.75in]{geometry}
\usepackage{graphicx}
\usepackage{hyperref}
\hypersetup{
	linkcolor=blue,
	colorlinks=true
}
\usepackage{blkarray}


\title{The theory and implementation of kCON}

\author{Xin Chen}
\date{\today}


\begin{document}
\maketitle


\section{Energy}

The energy of kCON is modeled under the many-body expression scheme:

\begin{eqnarray}
E^{total} = \sum_{a}^{C^N_1}{E^{(k=1)}_{a}} + \sum_{a,b}^{C^{N}_2}{E^{(k=2)}_{ab}} 
+ \sum_{a,b,c}^{C^{N}_3}{E^{(k=3)}_{abc}} + \cdots
\end{eqnarray}

\noindent For most cases, the equation above can be truncated at $k = 3$. Higher order 
terms contribute far less to the total energy while require much more computational
resources as $C^N_k$ roughly scales as $\mathcal{O}(N)$.

\subsection{One-body}

In kCON, the 1-body terms are expressed by a linear model:

\begin{eqnarray}
\sum_{a}^{C^N_1}{E^{(k=1)}_{a}} = \sum_{a}^{N}{E^{A_a}} = \sum_{A_{a}}{n^{A_a}E^{A_a}}
\end{eqnarray}

\noindent As an example, the 1-body contribution of a 
$\mathrm{C}_9 \mathrm{H}_7 \mathrm{N}$ is:

\begin{eqnarray}
E^{(k=1)} = 9E^{\mathrm{C}} + 7E^{\mathrm{H}} + E^{\mathrm{N}}
\end{eqnarray}

\noindent Here $E^{\mathrm{C}}$, $E^{\mathrm{H}}$ and $E^{\mathrm{N}}$ are all
trainable parameters.

\subsection{Two/three-body}

Independent convolutional neural networks are used to fit two/three-body terms:

\begin{eqnarray}
E^{(k=2)} & = & \sum_{a,b}^{C^N_2}{
	\boldmath{\mathrm{CNN}}^{\mathrm{A}_{a}\mathrm{A}_{b}}
}(r_{ab}) \\
E^{(k=3)} & = & \sum_{a,b,c}^{C^N_3}{
	\boldmath{\mathrm{CNN}}^{\mathrm{A}_{a}\mathrm{A}_{b}\mathrm{A}_{c}}
}(r_{ab}, r_{ac}, r_{bc})
\end{eqnarray}

\noindent To simplify the equations and normalize the distances $r$, an exponential 
transformation is adopted:

\begin{eqnarray}
z_{ab} 
= \exp{\left(-\frac{r_{ab}}{L_{a} + L_{b}}\right)}
= \exp{\left(-\frac{r_{ab}}{L_{ab}}\right)}
\end{eqnarray}

\noindent where $L_{a}$ is the covalent radius of element $A_{a}$. For non periodic 
molecules the P$\ddot{\mathrm{y}}\ddot{\mathrm{y}}$kko radii are used.

\subsection{The construction of the input feature matrix}

Now we can build the input feature matrix in a straightforward way. Taking the example 
of $\mathrm{C}_9 \mathrm{H}_7 \mathrm{N}$, the k-body terms and their dimensions are:

\begin{center}
	\scalebox{1.0}{
		\begin{tabular}{l*{3}{c}r}
			Term              & k & Expression & Dimension \\
			\hline
			CC    & 2 & $C^9_2$                         &  36 \\
			CH    & 2 & $C^9_1 \cdot C^7_1$             &  63 \\
			CN    & 2 & $C^9_1 \cdot C^1_1$             &   9 \\
			HH    & 2 & $C^7_2$                         &  21 \\
			HN    & 2 & $C^7_1 \cdot C^1_1$             &   7 \\
			CCC   & 3 & $C^9_3$                         &  84 \\
			CCH   & 3 & $C^9_2 \cdot C^7_1$             & 252 \\
			CCN   & 3 & $C^9_2 \cdot C^1_1$             &  36 \\
			CHH   & 3 & $C^9_1 \cdot C^7_2$             & 189 \\
			CHN   & 3 & $C^9_1 \cdot C^7_1 \cdot C^1_1$ &  63 \\
			HHH   & 3 & $C^7_3$                         &  35 \\
			HHN   & 3 & $C^7_2 \cdot C^1_1$             &  21 \\
			Total &   &                                 & 816 
		\end{tabular}	
	}
\end{center}

\noindent One can notice that $\mathrm{C}_9 \mathrm{H}_7 \mathrm{N}$ has 17 atoms but 
$816=C^{18}_3$. The CC and CCN, CH and CHN, HH and HHN all have the same dimension because $
\mathrm{C}_9 \mathrm{H}_7 \mathrm{N}$ only has one nitrogen atom. So here we propose a 'ghost 
atom' scheme which can simplify the construction of the input feature matrix. A ghost atom, 
denoted as X, is appended. Its distances to any real atom is $+\infty$. According to equation 
6, the scaled distance of X to any real atom will be zero. Then we drop all previous 2-body 
terms and only keep 3-body terms:

\begin{center}
	\scalebox{1.0}{
		\begin{tabular}{l*{3}{c}r}
			Term              & k & Expression & Dimension \\
			\hline
			CCX   & 3 & $C^9_2 \cdot C^1_1$             &  36 \\
			CHX   & 3 & $C^9_1 \cdot C^7_1 \cdot C^1_1$ &  63 \\
			CNX   & 3 & $C^9_1 \cdot C^1_1 \cdot C^1_1$ &   9 \\
			HHX   & 3 & $C^7_2 \cdot C^1_1$             &  21 \\
			HNX   & 3 & $C^7_1 \cdot C^1_1 \cdot C^1_1$ &   7 \\
			CCC   & 3 & $C^9_3$                         &  84 \\
			CCH   & 3 & $C^9_2 \cdot C^7_1$             & 252 \\
			CCN   & 3 & $C^9_2 \cdot C^1_1$             &  36 \\
			CHH   & 3 & $C^9_1 \cdot C^7_2$             & 189 \\
			CHN   & 3 & $C^9_1 \cdot C^7_1 \cdot C^1_1$ &  63 \\
			HHH   & 3 & $C^7_3$                         &  35 \\
			HHN   & 3 & $C^7_2 \cdot C^1_1$             &  21 \\
			Total &   &                                 & 816 
		\end{tabular}	
	}
\end{center}

\noindent Now we only have 3-body terms but the dimensions are unchanged. With this scheme, we 
can use just one matrix to store all 2-body and 3-body features.

\subsection{Permutational Invariance}

The three spatial (translational, rotational and permutational) invariances must all be 
satisfied. Since kCON only uses interatomic distances $r$, the translational and 
rotational invariances are naturally kept and the 1-body and 2-body terms are also 
permutationally invariant. However, the 3-body terms \(or higher\) do not satisfy this
requirement because the orders of the parameters of convolutional kernels are fixed 
while $(\mathrm{A}_{a}\mathrm{A}_{b}\mathrm{A}_{c})$ are inter-changeable. 

To overcome this problem, the conditional sorting scheme is adopted: the columns of the 
input matrix (the input layer) of each k-body CNN are ordered according to the bond types, 
and for each k-body interaction (matrix row) we only sort the  entries of the same atom 
types. Taking the examples of CHN, CCH and CCC:

\begin{itemize}
	\item CHN: each column represents a unique bond type (C-H, C-N, H-N). Sorting is not needed.
	\item CCN: the entries corresponding to C-C of each row should be sorted.
	\item CCC: all three entries of each row should be sorted.
\end{itemize}

\subsection{Loss}

By default, kCON uses the root mean squared error as the total loss to minimize:

\begin{eqnarray}
\mathrm{RMSE} = \sqrt{
	\frac{1}{n}
	\sum_{i=1}^{n}{ 
		\left( y_{i}^{\mathrm{kCON}} - y_{i}^{\mathrm{True}} \right)^2
	}
}	
\end{eqnarray}

\noindent kCON also supports exponentially-scaled RMSE if the contributions of the unstable 
structures should be minimized:

\begin{eqnarray}
\mathrm{esRMSE} = \sqrt{
	\frac{1}{n} 
	\sum_{i=1}^{n}{
		\exp{\left(-\frac{y_{i}^{\mathrm{kCON}} - y_{i}^{\mathrm{True}}}{k_BT} \right)}
		\left(y_{i}^{\mathrm{kCON}} - y_{i}^{\mathrm{True}} \right)^2
	}
}	
\end{eqnarray}

\section{Theoretical analysis of the forces}

Atomic force is the negative first-order derivative of energy with respect to 
displacement:

\begin{eqnarray}
f(r) = -\frac{\partial E(r)}{\partial r}
\end{eqnarray}

\noindent So it's straightforward to get kCON atomic forces:

\begin{eqnarray}
f(x_{i}) & = & -\frac{\partial{E^{total}}}{\partial{x_{i}}} \nonumber \\
& = & -\left(
	\frac{\partial{E^{(k=1)}}}{\partial{x_i}} 
	+ \frac{\partial{E^{(k=2)}}}{\partial{x_i}}
	+ \frac{\partial{E^{(k=3)}}}{\partial{x_i}} 
\right) \nonumber \\
& = & -\left(
\frac{
	\partial{\sum_{a,b}^{C^N_2}{
		\boldmath{\mathrm{CNN}}^{\mathrm{A}_{a}\mathrm{A}_{b}}}(z_{ab})}
	}{
		\partial{x_i}
	} 
+ 
\frac{
	\partial{\sum_{a,b,c}^{C^N_3}{
		\boldmath{\mathrm{CNN}}^{\mathrm{A}_{a}\mathrm{A}_{b}\mathrm{A}_{c}}}
		(z_{ab}, z_{ac}, z_{bc})}
	}{
		\partial{x_i}
	} 
\right) \nonumber \\
& = & -\left(
\sum_{a,b}^{C^N_2}{
	\frac{
		\partial{\boldmath{\mathrm{CNN}}^{\mathrm{A}_{a}\mathrm{A}_{b}}(z_{ab})}
	}{
		\partial{x_i}
	}
}
	+
\sum_{a,b,c}^{C^N_3}{
	\frac{
		\partial{\boldmath{\mathrm{CNN}}^{\mathrm{A}_{a}\mathrm{A}_{b}\mathrm{A}_{c}}
		(z_{ab}, z_{ac}, z_{bc})}
	}{
		\partial{x_i}
	}
}
\right)
\end{eqnarray}

\noindent where $f(x_{i})$ is the force component of atom $i$ along the X direction. We 
also have:

\begin{eqnarray}
%
% equation 11
%
\frac{
	\partial{\boldmath{\mathrm{CNN}}^{\mathrm{A}_{a}\mathrm{A}_{b}}(z_{ab})}
}{
	\partial{x_i}
} 
& = & 
\frac{\partial{
	\boldmath{\mathrm{CNN}}^{\mathrm{A}_{a}\mathrm{A}_{b}}(z_{ab})}
}{
	\partial{z_{ab}}
} \frac{\partial{z_{ab}}}{\partial{r_{ab}}} \frac{\partial{r_{ab}}}{\partial{x_i}} \\
%
% equation 12
%
\frac{ \partial{z_{ab}}}{\partial{r_{ab}}} 
& = & 
-\frac{z_{ab}}{L_{A_a} + L_{A_b}} = -\frac{z_{ab}}{L_{ab}} \\
%
% equation 13
%
\frac{\partial{r_{ab}}}{\partial{x_i}} & = & \begin{cases}
	\frac{x_a - x_b}{r_{ab}} & \quad \text{if } i = a \\
	-\frac{x_a - x_b}{r_{ab}} & \quad \text{if } i = b \\
	0                        & \quad \text{else}
\end{cases} 
\end{eqnarray}

\noindent Finally we get:

\begin{eqnarray}
r_{ab} & = & -L_{ab}\log{\left( z_{ab} \right)} \\
d^x_{ab} & = & x_{a} - x_{b} \\
d^y_{ab} & = & y_{a} - y_{b} \\
d^z_{ab} & = & z_{a} - z_{b} \\
\frac{\partial{z_{ab}}}{\partial{r_{ab}}} \frac{\partial{r_{ab}}}{\partial{\{x, y, z\}_i}} 
& = &
\begin{cases}
z_{ab} d^{\{x,y,z\}}_{ab} / (L_{ab}^{2} \log{(z_{ab})}) & \quad \text{if } i = a \\
-z_{ab} d^{\{x,y,z\}}_{ab} / (L_{ab}^{2} \log{(z_{ab})}) & \quad \text{if } i = b \\
0 & \quad \text{else}
\end{cases}
\end{eqnarray}

For three (or higher body) terms, the result is similar to Equation 11 (See the Appendix 
A for detailed derivation):

\begin{eqnarray}
\frac{
	\partial{\boldmath{\mathrm{CNN}}^{k}\left(\left\{ z \right\} \right)}
}{
	\partial{x_i}
} 
=  \sum_{ab}^{C^N_k}{
\frac{
	\partial{\boldmath{\mathrm{CNN}}^{k}\left(\left\{ z \right\} \right)}
}{
	\partial{z_{ab}}
} \frac{\partial{z_{ab}}}{\partial{r_{ab}}}\frac{\partial{r_{ab}}}{\partial{x_i}}}
\end{eqnarray}

As kCON is built upon Google's TensorFlow, the calculations of the gradients above become 
far more easier as TensorFlow can output the these complicated derivatives automatically:

\begin{eqnarray}
\frac{
	\partial{\boldmath{\mathrm{CNN}}^{k}\left(\left\{ z \right\} \right)}
}{
	\partial{z_{ab}}
}	
\end{eqnarray}

\section{Implementation of the forces}

The implementation the atomic forces is a complicated though the theoretical analysis is 
clear because we must make all operations \textbf{vectorizable} so that we can take
advantages of modern deep learning frameworks like TensorFlow or MXNet. 

\subsection{Dimension analysis}

Suppose we have a system composed of N atoms with $k^\mathrm{max}=3$, the total energy can be 
computed with the following equations:

\begin{eqnarray}
	E^{\mathrm{kCON}} & = & E^{(k=1)} + \mathrm{NN}(\boldsymbol{Z}) \\
	\boldsymbol{Z} & = & \left[
		\begin{array}{c}
			\vec{z_{1}}  \\
			\vec{z_{2}}  \\
			\vec{z_{3}}  \\
			\vec{z_{4}}  \\
			\vec{z_{5}}  \\
			\vdots \\
			\vec{z_{n}}  \\
		\end{array}
	\right]                                                         \\
	n & = & C^{N + 1}_3                                             \\
	\boldsymbol{L} & = & \left[
		\begin{array}{c}
			\vec{l_{1}}  \\
			\vec{l_{2}}  \\
			\vec{l_{3}}  \\
			\vec{l_{4}}  \\
			\vec{l_{5}}  \\
			\vdots \\
			\vec{l_{n}}  \\			
		\end{array}
	\right]
\end{eqnarray}

\noindent where \textbf{Z} is the input feature matrix with shape $[C^{N+1}_{3}, 3]$, 
$\vec{z_{i}}$ is a three-components vector representing the \textbf{conditionally sorted} 
features of a chemical pattern and \textbf{L} is the associated covalent radii matrix for 
\textbf{Z}. $E^{(k=1)}$ does not depend on interatomic distances, so we can safely ignore it 
when computing atomic forces. Now TensorFlow can output the derivatives of 
$E^{\mathrm{kCON}}$ with respect to the input feature matrix \textbf{Z} directly:

\begin{equation}
\frac{\partial{E^{\mathrm{kCON}}}}{\partial{\mathbf{Z}}} = 
\left[
	\begin{array}{c}
		\partial{E} / \partial{\vec{z_{1}}}  \\
		\partial{E} / \partial{\vec{z_{2}}}  \\
		\partial{E} / \partial{\vec{z_{3}}}  \\
		\partial{E} / \partial{\vec{z_{4}}}  \\
		\partial{E} / \partial{\vec{z_{5}}}  \\
		\vdots \\
		\partial{E} / \partial{\vec{z_{n}}}  \\
	\end{array}
\right]
\end{equation}

\noindent and the shape of $\partial{E^{\mathrm{kCON}}} / \partial{\boldsymbol{Z}}$ is
also $[C^{N+1}_{3}, 3]$. 

Now let's look into $\partial{E} / \partial{\vec{z_{i}}}$. Here we define 
$\vec{z_{1}} = [z_{12}, z_{13}, z_{23}]$ where $z_{ab}$ is the scaled interatomic distance of
atom $a$ and $b$. According to equation 15, $\partial{z_{12}} / \partial{x_{i}}$ will be non
-zero if and only if $i = a$ or $i = b$. Hence, 
$\partial{E} / \partial{z_{12}} \cdot \partial{z_{12}} / \partial{x_{i}}$ will only give 
effective contributions to \textbf{six} atomic force components: $f^x_1$, $f^y_1$, $f^z_1$, 
$f^x_2$, $f^y_2$ and $f^z_2$. Thus, $\partial{E^{\mathrm{kCON}}} / \partial{\boldsymbol{Z}}$ 
will produce $6 \cdot C^{N+1}_3 \cdot 3=18C^{N+1}_3$ atomic force contributions but only
$6 \cdot C^N_3 \cdot 3 + 6 \cdot C^N_2 \cdot 1$ of them are effective because the ghost atom 
should give zero contribution. Since we have N atoms, there will be 3N force components and 
each force component is the sum of $(N - 1)^2$ force contributions.

\subsection{Tiling}

According to the dimension analysis, each entry of \textbf{Z}, \textbf{L} and 
$\partial{E^{\mathrm{kCON}}} / \partial{Z}$ corresponds to six force components. So repeating 
these matrices 6 times will make each entry correspond to only one force component. This can
be achieved by  
\href{https://docs.scipy.org/doc/numpy/reference/generated/numpy.tile.html}{tiling}.
The following example demonstrates the tiled \textbf{Z}:

\begin{equation}
Z_{tiled} = \mathrm{tile}(Z, (1,6)) = \left[
\begin{array}{cccccc}
	\vec{z_{1}} & \vec{z_{1}} & \vec{z_{1}} & \vec{z_{1}} & \vec{z_{1}} & \vec{z_{1}}  \\
	\vec{z_{2}} & \vec{z_{2}} & \vec{z_{2}} & \vec{z_{2}} & \vec{z_{2}} & \vec{z_{2}}  \\
	\vec{z_{3}} & \vec{z_{3}} & \vec{z_{3}} & \vec{z_{3}} & \vec{z_{3}} & \vec{z_{3}}  \\
	\vec{z_{4}} & \vec{z_{4}} & \vec{z_{4}} & \vec{z_{4}} & \vec{z_{4}} & \vec{z_{4}}  \\
	\vec{z_{5}} & \vec{z_{5}} & \vec{z_{5}} & \vec{z_{5}} & \vec{z_{5}} & \vec{z_{5}}  \\
	\vdots      & \vdots      & \vdots      & \vdots      & \vdots      & \vdots       \\
	\vec{z_{n}} & \vec{z_{n}} & \vec{z_{n}} & \vec{z_{n}} & \vec{z_{n}} & \vec{z_{n}}  \\
		\end{array}
	\right]
\end{equation}

\noindent After the tiling, the shapes of $\boldsymbol{Z}_{tiled}$, $\boldsymbol{L}_{tiled}$ 
and $(\partial{E^{\mathrm{kCON}}} / \partial{Z})_{tiled}$ now become $[C^{N+1}_{3}, 18]$.

\subsection{Coordinates differences}

The one last auxiliary matrix to compute is the differences of the atomic coordinates 
$d^{\{x, y, z\}}_{ab}$ introduced in equation 18. The matrix, denoted as \textbf{D}, also has 
the shape of $[C^{N+1}_{3}, 18]$:

\begin{equation}
\mathbf{D} = \left[ 
\begin{array}{ccccccc}
\vec{d}_1 & \vec{d}_2 & \vec{d}_3 & \vec{d}_4 & \vec{d}_5 & \dots & \vec{d}_n 
\end{array}
\right]^T
\end{equation}

Suppose $\vec{z}_{1} = [z_{12}, z_{13}, z_{23}]$ and 
$\vec{l}_{1} = [l_{12}, l_{13}, l_{23}]$. After the tiling, we have:
\begin{eqnarray}
\left(\vec{z}_{1, tiled}\right)^T = \left[
	\begin{array}{c}
		z_{12} \\
		z_{13} \\
		z_{23} \\
		z_{12} \\
		z_{13} \\
		z_{23} \\
		z_{12} \\
		z_{13} \\
		z_{23} \\
		z_{12} \\
		z_{13} \\
		z_{23} \\
		z_{12} \\
		z_{13} \\
		z_{23} \\
		z_{12} \\
		z_{13} \\
		z_{23} \\
	\end{array}
\right]
, \quad
\left(\vec{l}_{1, tiled}\right)^T = \left[
	\begin{array}{c}
		l_{12} \\
		l_{13} \\
		l_{23} \\
		l_{12} \\
		l_{13} \\
		l_{23} \\
		l_{12} \\
		l_{13} \\
		l_{23} \\
		l_{12} \\
		l_{13} \\
		l_{23} \\
		l_{12} \\
		l_{13} \\
		l_{23} \\
		l_{12} \\
		l_{13} \\
		l_{23} \\
	\end{array}
\right]
\end{eqnarray}

\noindent So, we can easily compute the corresponding $d^{\{ x,y,z \}}_{ab}$:

\begin{equation}
\vec{d}_{1} = \begin{blockarray}{cc}
              & component \\
\begin{block}{(c)c}
	+d^x_{12} & f^x_1 \\
	+d^x_{13} & f^x_1 \\
	+d^x_{23} & f^x_2 \\
	+d^y_{12} & f^y_1 \\
	+d^y_{13} & f^y_1 \\
	+d^y_{23} & f^y_2 \\
	+d^z_{12} & f^z_1 \\
	+d^z_{13} & f^z_1 \\
	+d^z_{23} & f^z_2 \\
	-d^x_{12} & f^x_2 \\
	-d^x_{13} & f^x_3 \\
	-d^x_{23} & f^x_3 \\
	-d^y_{12} & f^y_2 \\
	-d^y_{13} & f^y_3 \\
	-d^y_{23} & f^y_3 \\
	-d^z_{12} & f^z_2 \\
	-d^z_{13} & f^z_3 \\
	-d^z_{23} & f^z_3 \\
\end{block}
\end{blockarray}
\end{equation}

\subsection{Atomic forces}

Finally we can compute atomic forces.



\newpage

\section*{Appendix: an example of $\mathrm{B}_4$}

Consider a simple example of $\mathrm{B}_{4}$. This molecule has 4 atoms, so the total 
dimension of the input feature matrix should be $C^{4+1}_3=10$, including $C^4_3$ 3-body 
features (BBB) and $C^4_2$ 2-body features (BBX).
Its input feature matrix, $\mathbf{Z}^{(0)}$, can be expressed like this:

\begin{eqnarray}
\mathbf{Z}^{(0)} & = & \left[\begin{array}{c}
	\mathbf{Z^{(0)}}_{\mathrm{BBX}} \\
	\mathbf{Z^{(0)}}_{\mathrm{BBB}} \\
\end{array}
\right]
\end{eqnarray}

\noindent where:

\begin{eqnarray}
f(r_{ab}) & = & \exp{\left( -\frac{r_{ab}}{L_{ab}} \right)} \\
\mathbf{Z}^{(0)}_{\mathrm{BBX}} & = & \begin{blockarray}{ccc}
  \mathrm{BB} & \mathrm{BX} & \mathrm{BX} \\
\begin{block}{(ccc)}
	z_{11} & 0        & 0        \\
	z_{21} & 0        & 0        \\
	z_{31} & 0        & 0        \\
	z_{41} & 0        & 0        \\
	z_{51} & 0        & 0        \\
	z_{61} & 0        & 0        \\
\end{block}
\end{blockarray} = 
f(\left[\begin{array}{ccc}
	r_{12} & +\infty  & +\infty  \\
	r_{13} & +\infty  & +\infty  \\
	r_{14} & +\infty  & +\infty  \\
	r_{23} & +\infty  & +\infty  \\
	r_{24} & +\infty  & +\infty  \\
	r_{34} & +\infty  & +\infty  \\
\end{array}
\right]) \\
\mathbf{Z}^{(0)}_{\mathrm{BBB}} & = & \begin{blockarray}{ccc}
\mathrm{BB} & \mathrm{BB} & \mathrm{BB} \\ 
\begin{block}{(ccc)}
	z_{11} & z_{12}   & z_{13}   \\
	z_{21} & z_{22}   & z_{23}   \\
	z_{31} & z_{32}   & z_{33}   \\
	z_{41} & z_{42}   & z_{43}   \\
\end{block}
\end{blockarray} =
f(\left[\begin{array}{ccc}
	r_{12} & r_{13}   & r_{23}   \\
	r_{12} & r_{14}   & r_{24}   \\
	r_{13} & r_{14}   & r_{34}   \\
	r_{23} & r_{24}   & r_{34}   \\
\end{array}
\right])
\end{eqnarray}

\noindent The analysis will only consider the 3-body part for simplicity.

Suppose the convolutional neural network for BBB has two hidden layers and each hidden 
layer has two kernels, then we can get the convolutional kernels:

\begin{eqnarray}
\mathbf{W}^{(1)} & = & \left[\begin{array}{ccc}
	w^{(1)}_{11} & w^{(1)}_{12} & w^{(1)}_{13} \\
	w^{(1)}_{21} & w^{(1)}_{22} & w^{(1)}_{23} \\
\end{array}
\right] \\
\mathbf{b}^{(1)} & = & \left[\begin{array}{c}
	b^{(1)}_{1} \\
	b^{(1)}_{2} \\
\end{array}
\right] \\
\mathbf{W}^{(2)} & = & \left[\begin{array}{cc}
	w^{(2)}_{11} & w^{(2)}_{12} \\
	w^{(2)}_{21} & w^{(2)}_{22} \\
\end{array}
\right] \\
\mathbf{b}^{(2)} & = & \left[\begin{array}{c}
	b^{(2)}_{1} \\
	b^{(2)}_{2} \\
\end{array}
\right] \\
\mathbf{W}^{(3)} & = & \left[\begin{array}{cc}
	w^{(3)}_{11} & w^{(3)}_{12} \\
\end{array}
\right]
\end{eqnarray}

\noindent where $\mathbf{W}^{(l)}$ is the kernel matrix for layer $l$ and each row of  
$\mathbf{W}^{(l)}$ represents a kernel. $\mathbf{b}^{(l)}$ are the biases for the kernels of 
layer $l$. One should notice that the last layer (the output layer) does not have a bias unit.
The activation function $\sigma(\cdot)$ used here is leaky ReLU:

\begin{eqnarray}
\sigma(x) & = & \begin{cases}
	x & \quad \text{if } x \geq 0 \\
	\alpha x & \quad \text{else} \\
\end{cases} \\
\frac{\partial{\sigma(x)}}{\partial{x}} & = & \begin{cases}
	1 & \quad \text{if } x \geq 0 \\
	\alpha & \quad \text{else} \\
\end{cases} \\
\alpha & = & 0.2
\end{eqnarray}

\noindent Now we can start the forward propagation. The results of the first layer,
$\mathbf{Z}^{(1)}$, should be:

\begin{eqnarray}
\mathbf{Z}^{(1)} 
& = & \left[\begin{array}{ccc}
	\sigma(w^{(1)}_{11}z_{11} + w^{(1)}_{12}z_{12} + w^{(1)}_{13}z_{13} + b^{(1)}_1) &
	\sigma(w^{(1)}_{21}z_{11} + w^{(1)}_{22}z_{12} + w^{(1)}_{23}z_{13} + b^{(1)}_2) \\
	\sigma(w^{(1)}_{11}z_{21} + w^{(1)}_{12}z_{22} + w^{(1)}_{13}z_{23} + b^{(1)}_1) &
	\sigma(w^{(1)}_{21}z_{21} + w^{(1)}_{22}z_{22} + w^{(1)}_{23}z_{23} + b^{(1)}_2) \\
	\sigma(w^{(1)}_{11}z_{31} + w^{(1)}_{12}z_{32} + w^{(1)}_{13}z_{33} + b^{(1)}_1) &
	\sigma(w^{(1)}_{21}z_{31} + w^{(1)}_{22}z_{32} + w^{(1)}_{23}z_{33} + b^{(1)}_2) \\
	\sigma(w^{(1)}_{11}z_{41} + w^{(1)}_{12}z_{42} + w^{(1)}_{13}z_{43} + b^{(1)}_1) &
	\sigma(w^{(1)}_{21}z_{41} + w^{(1)}_{22}z_{42} + w^{(1)}_{23}z_{43} + b^{(1)}_2) \\
\end{array}
\right] \nonumber \\
& = & \left[\begin{array}{ccc}	
	\sigma(a^{(1)}_{11}) & \sigma(a^{(1)}_{12}) \\
	\sigma(a^{(1)}_{21}) & \sigma(a^{(1)}_{22}) \\
	\sigma(a^{(1)}_{31}) & \sigma(a^{(1)}_{32}) \\
	\sigma(a^{(1)}_{41}) & \sigma(a^{(1)}_{42}) \\
\end{array}
\right] \nonumber \\
& = & \left[\begin{array}{ccc}	
	z^{(1)}_{11} & z^{(1)}_{12} \\
	z^{(1)}_{21} & z^{(1)}_{22} \\
	z^{(1)}_{31} & z^{(1)}_{32} \\
	z^{(1)}_{41} & z^{(1)}_{42} \\
\end{array}
\right]
\end{eqnarray}

\noindent Then we can calculate the results of the second layer, $\mathbf{Z}^{(2)}$:

\begin{eqnarray}
\mathbf{Z}^{(2)} 
& = & \left[\begin{array}{ccc}
	\sigma(w^{(2)}_{11}z^{(1)}_{11} + w^{(2)}_{12}z^{(1)}_{12} + b^{(2)}_1) &
	\sigma(w^{(2)}_{21}z^{(1)}_{11} + w^{(2)}_{22}z^{(1)}_{12} + b^{(2)}_2) \\
	\sigma(w^{(2)}_{11}z^{(1)}_{21} + w^{(2)}_{12}z^{(1)}_{22} + b^{(2)}_1) &
	\sigma(w^{(2)}_{21}z^{(1)}_{21} + w^{(2)}_{22}z^{(1)}_{22} + b^{(2)}_2) \\
	\sigma(w^{(2)}_{11}z^{(1)}_{31} + w^{(2)}_{12}z^{(1)}_{32} + b^{(2)}_1) &
	\sigma(w^{(2)}_{21}z^{(1)}_{31} + w^{(2)}_{22}z^{(1)}_{32} + b^{(2)}_2) \\
	\sigma(w^{(2)}_{11}z^{(1)}_{41} + w^{(2)}_{12}z^{(1)}_{42} + b^{(2)}_1) &
	\sigma(w^{(2)}_{21}z^{(1)}_{41} + w^{(2)}_{22}z^{(1)}_{42} + b^{(2)}_2) \\
\end{array}
\right] \nonumber \\
& = & \left[\begin{array}{ccc}	
	\sigma(a^{(2)}_{11}) & \sigma(a^{(2)}_{12}) \\
	\sigma(a^{(2)}_{21}) & \sigma(a^{(2)}_{22}) \\
	\sigma(a^{(2)}_{31}) & \sigma(a^{(2)}_{32}) \\
	\sigma(a^{(2)}_{41}) & \sigma(a^{(2)}_{42}) \\
\end{array}
\right] \nonumber \\
& = & \left[\begin{array}{ccc}	
	z^{(2)}_{11} & z^{(2)}_{12} \\
	z^{(2)}_{21} & z^{(2)}_{22} \\
	z^{(2)}_{31} & z^{(2)}_{32} \\
	z^{(2)}_{41} & z^{(2)}_{42} \\
\end{array}
\right]
\end{eqnarray}

\noindent The results of output layer, $\mathbf{Z}^{(3)}$, should be:

\begin{equation}
\mathbf{Z}^{(3)} = \left[\begin{array}{c}
	z^{(2)}_{11}w^{(3)}_{11} + z^{(2)}_{12}w^{(3)}_{12} \\
	z^{(2)}_{21}w^{(3)}_{11} + z^{(2)}_{22}w^{(3)}_{12} \\
	z^{(2)}_{31}w^{(3)}_{11} + z^{(2)}_{32}w^{(3)}_{12} \\
	z^{(2)}_{41}w^{(3)}_{11} + z^{(2)}_{42}w^{(3)}_{12} \\
\end{array}
\right]
\end{equation}

\noindent The activation function will not be applied to the output layer.
Each entry of $\mathbf{Z}^{(3)}$ represents the k-body energy of its corresponding input 
chemical pattern (row of the matrix) of $\mathrm{Z}^{(0)}$.
The total energy $E$ is just the sum of the entries of $\mathbf{Z}^{(3)}$:

\begin{eqnarray}
E 
& = & 
z^{(2)}_{11}w^{(3)}_{11} + z^{(2)}_{12}w^{(3)}_{12} + 
z^{(2)}_{21}w^{(3)}_{11} + z^{(2)}_{22}w^{(3)}_{12} + 
z^{(2)}_{31}w^{(3)}_{11} + z^{(2)}_{32}w^{(3)}_{12} + 
z^{(2)}_{41}w^{(3)}_{11} + z^{(2)}_{42}w^{(3)}_{12} \nonumber \\
& = &
\left(\begin{array}{cccccccc}
	1 & 1 & 1 & 1 & 1 & 1 & 1 & 1 \\
\end{array}
\right)^\mathbf{T}
\left[\begin{array}{c}
	z^{(2)}_{11}w^{(3)}_{11} \\
	z^{(2)}_{12}w^{(3)}_{12} \\
	z^{(2)}_{21}w^{(3)}_{11} \\
	z^{(2)}_{22}w^{(3)}_{12} \\
	z^{(2)}_{31}w^{(3)}_{11} \\
	z^{(2)}_{32}w^{(3)}_{12} \\
	z^{(2)}_{41}w^{(3)}_{11} \\
	z^{(2)}_{42}w^{(3)}_{12} \\
\end{array}
\right] \nonumber \\
& = &
\mathbf{I}^\mathbf{T}
\left[\begin{array}{c}
	\sigma(w^{(2)}_{11}z^{(1)}_{11} + w^{(2)}_{12}z^{(1)}_{12} + b^{(2)}_1)w^3_{11} \\
	\sigma(w^{(2)}_{21}z^{(1)}_{11} + w^{(2)}_{22}z^{(1)}_{12} + b^{(2)}_2)w^3_{12} \\
	\sigma(w^{(2)}_{11}z^{(1)}_{21} + w^{(2)}_{12}z^{(1)}_{22} + b^{(2)}_1)w^3_{11} \\ 
	\sigma(w^{(2)}_{21}z^{(1)}_{21} + w^{(2)}_{22}z^{(1)}_{22} + b^{(2)}_2)w^3_{12} \\
	\sigma(w^{(2)}_{11}z^{(1)}_{31} + w^{(2)}_{12}z^{(1)}_{32} + b^{(2)}_1)w^3_{11} \\
	\sigma(w^{(2)}_{21}z^{(1)}_{31} + w^{(2)}_{22}z^{(1)}_{32} + b^{(2)}_2)w^3_{12} \\
	\sigma(w^{(2)}_{11}z^{(1)}_{41} + w^{(2)}_{12}z^{(1)}_{42} + b^{(2)}_1)w^3_{11} \\
	\sigma(w^{(2)}_{21}z^{(1)}_{41} + w^{(2)}_{22}z^{(1)}_{42} + b^{(2)}_2)w^3_{12}
\end{array}
\right] \nonumber \\ 
& = & 
\mathbf{I}^\mathbf{T} \left[\begin{array}{c}
	Y_{1} \\
	Y_{2} \\
	Y_{3} \\
	Y_{4} \\
	Y_{5} \\
	Y_{6} \\
	Y_{7} \\
	Y_{8} \\
\end{array}
\right] \nonumber \\
& = & 
\mathbf{I}^\mathbf{T} \mathbf{Y}
\end{eqnarray}

\noindent where \textbf{Y} is:

\begin{equation}
\mathbf{Y} = 
\left[\begin{array}{c}
	\sigma\left(
		w^{(2)}_{11}
			\sigma(w^{(1)}_{11}z_{11} + w^{(1)}_{12}z_{12} + w^{(1)}_{13}z_{13} + b^{(1)}_1) + 
		w^{(2)}_{12}
			\sigma(w^{(1)}_{21}z_{11} + w^{(1)}_{22}z_{12} + w^{(1)}_{23}z_{13} + b^{(1)}_2) + 
		b^{(2)}_1
	\right)w^{(3)}_{11} \\
	\sigma\left(
		w^{(2)}_{21}
			\sigma(w^{(1)}_{11}z_{11} + w^{(1)}_{12}z_{12} + w^{(1)}_{13}z_{13} + b^{(1)}_1) + 
		w^{(2)}_{22}
			\sigma(w^{(1)}_{21}z_{11} + w^{(1)}_{22}z_{12} + w^{(1)}_{23}z_{13} + b^{(1)}_2) + 
		b^{(2)}_2
	\right)w^{(3)}_{12} \\
	\sigma\left(
		w^{(2)}_{11}
			\sigma(w^{(1)}_{11}z_{21} + w^{(1)}_{12}z_{22} + w^{(1)}_{13}z_{23} + b^{(1)}_1) + 
		w^{(2)}_{12}
			\sigma(w^{(1)}_{21}z_{21} + w^{(1)}_{22}z_{22} + w^{(1)}_{23}z_{23} + b^{(1)}_2) + 
		b^{(2)}_1
	\right)w^{(3)}_{11} \\ 
	\sigma\left(
		w^{(2)}_{21}
			\sigma(w^{(1)}_{11}z_{21} + w^{(1)}_{12}z_{22} + w^{(1)}_{13}z_{23} + b^{(1)}_1) + 
		w^{(2)}_{22}
			\sigma(w^{(1)}_{21}z_{21} + w^{(1)}_{22}z_{22} + w^{(1)}_{23}z_{23} + b^{(1)}_2) + 
		b^{(2)}_2
	\right)w^{(3)}_{12} \\
	\sigma\left(
		w^{(2)}_{11}
			\sigma(w^{(1)}_{11}z_{31} + w^{(1)}_{12}z_{32} + w^{(1)}_{13}z_{33} + b^{(1)}_1) + 
		w^{(2)}_{12}
			\sigma(w^{(1)}_{21}z_{31} + w^{(1)}_{22}z_{32} + w^{(1)}_{23}z_{33} + b^{(1)}_2) + 
		b^{(2)}_1
	\right)w^{(3)}_{11} \\
	\sigma\left(
		w^{(2)}_{21}
			\sigma(w^{(1)}_{11}z_{31} + w^{(1)}_{12}z_{32} + w^{(1)}_{13}z_{33} + b^{(1)}_1) +
		w^{(2)}_{22}
			\sigma(w^{(1)}_{21}z_{31} + w^{(1)}_{22}z_{32} + w^{(1)}_{23}z_{33} + b^{(1)}_2) + 
		b^{(2)}_2
	\right)w^{(3)}_{12} \\
	\sigma\left(
		w^{(2)}_{11}
			\sigma(w^{(1)}_{11}z_{41} + w^{(1)}_{12}z_{42} + w^{(1)}_{13}z_{43} + b^{(1)}_1) +
		w^{(2)}_{12}
			\sigma(w^{(1)}_{21}z_{41} + w^{(1)}_{22}z_{42} + w^{(1)}_{23}z_{43} + b^{(1)}_2) + 
		b^{(2)}_1
	\right)w^{(3)}_{11} \\
	\sigma\left(
		w^{(2)}_{21}
			\sigma(w^{(1)}_{11}z_{41} + w^{(1)}_{12}z_{42} + w^{(1)}_{13}z_{43} + b^{(1)}_1) + 
		w^{(2)}_{22}
			\sigma(w^{(1)}_{21}z_{41} + w^{(1)}_{22}z_{42} + w^{(1)}_{23}z_{43} + b^{(1)}_2) + 
		b^{(2)}_2
	\right)w^{(3)}_{12}
\end{array}
\right]
\end{equation}

Now the forward propagation is finished and the output (total energy) is obtained. Then we can 
start the back propagation. The calculation of atomic forces can be done at the same time.
To compute the atomic forces, we should first resolve the derivative of $E$ with respect to 
$z_{ab}$. Taking the example of $\partial{E} / \partial{z_{11}}$, we have:

\begin{eqnarray}
\frac{\partial{E}}{\partial{z_{11}}} 
& = &
\sum_{i=1}^8{
	\frac{\partial{Y_i}}{\partial{z_{11}}}
} \nonumber \\
& = &
\frac{\partial{Y_1}}{\partial{z_{11}}} + \frac{\partial{Y_2}}{\partial{z_{11}}} \\
& = & 
w^{(3)}_{11}\frac{\partial{\sigma(a_{11}^{(2)}})}{\partial{a_{11}^{(2)}}} 
\left(
	w_{11}^{(2)}\frac{\partial{\sigma(a_{11}^{(1)}})}{\partial{a_{11}^{(1)}}}w^{(1)}_{11} +
	w_{12}^{(2)}\frac{\partial{\sigma(a_{12}^{(1)}})}{\partial{a_{12}^{(1)}}}w^{(1)}_{21}  
\right) + \nonumber \\
&& 
w^{(3)}_{12}\frac{\partial{\sigma(a_{12}^{(2)}})}{\partial{a_{12}^{(2)}}} 
\left(
	w_{21}^{(2)}\frac{\partial{\sigma(a_{11}^{(1)}})}{\partial{a_{11}^{(1)}}}w^{(1)}_{11} +
	w_{22}^{(2)}\frac{\partial{\sigma(a_{12}^{(1)}})}{\partial{a_{12}^{(1)}}}w^{(1)}_{21}  
\right)
\end{eqnarray} 

\noindent Similarly, we can compute the derivatives of $E$ with respect to all entries of 
$\mathbf{Z}^{(0)}_{\mathrm{BBB}}$. Then we can calculate the derivative of $E$ with respect 
to an arbitrary force component, e.g. $x_1$:

\begin{equation}\label{dE_dYdz_dzdx}
\frac{\partial{E}}{\partial{x_1}} = \sum_{i=1}^{8}{
	\sum_{a,b}{
		\frac{\partial{Y_{i}}}{\partial{z_{ab}}} 
		\cdot 
		\frac{\partial{z_{ab}}}{\partial{x_{1}}}
	}
}
\end{equation}

\noindent Remember that:

\begin{eqnarray}
\left[\begin{array}{ccc}
	z_{11} & z_{12}   & z_{13}   \\
	z_{21} & z_{22}   & z_{23}   \\
	z_{31} & z_{32}   & z_{33}   \\
	z_{41} & z_{42}   & z_{43}   \\
\end{array}\right] =
f(\left[\begin{array}{ccc}
	r_{12} & r_{13}   & r_{23}   \\
	r_{12} & r_{14}   & r_{24}   \\
	r_{13} & r_{14}   & r_{34}   \\
	r_{23} & r_{24}   & r_{34}   \\
\end{array}
\right]) = f(\mathbf{R})
\end{eqnarray}

\noindent and:

\begin{eqnarray}
r_{ij} = \sqrt{(x_i - x_j)^2 + (y_i - y_j)^2 + (z_i - z_j)^2}
\end{eqnarray}

\noindent Only $\partial{z_{11}} / \partial{x_1}$, 
$\partial{z_{12}} / \partial{x_1}$, $\partial{z_{21}} / \partial{x_1}$,
$\partial{z_{22}} / \partial{x_1}$, $\partial{z_{31}} / \partial{x_1}$ and
$\partial{z_{32}} / \partial{x_1}$ are non-zero. Thus, equation \ref{dE_dYdz_dzdx} 
can be further simplified:

\begin{eqnarray}
\frac{\partial{E}}{\partial{x_1}} 
& = & 
\sum_{i=1}^{8}{
	\sum_{a,b}{
		\frac{\partial{Y_{i}}}{\partial{z_{ab}}} 
		\cdot 
		\frac{\partial{z_{ab}}}{\partial{x_{1}}}
	}
} \nonumber \\
& = & 
\sum_{i=1}^{8}{\left(
	\frac{\partial{Y_{i}}}{\partial{z_{11}}}\frac{\partial{z_{11}}}{\partial{x_1}} +
	\frac{\partial{Y_{i}}}{\partial{z_{12}}}\frac{\partial{z_{12}}}{\partial{x_1}} +
	\frac{\partial{Y_{i}}}{\partial{z_{21}}}\frac{\partial{z_{21}}}{\partial{x_1}} +
	\frac{\partial{Y_{i}}}{\partial{z_{22}}}\frac{\partial{z_{22}}}{\partial{x_1}} +
	\frac{\partial{Y_{i}}}{\partial{z_{31}}}\frac{\partial{z_{31}}}{\partial{x_1}} +
	\frac{\partial{Y_{i}}}{\partial{z_{32}}}\frac{\partial{z_{32}}}{\partial{x_1}}
\right)
} \nonumber \\
& = &
\left(
	\frac{\partial{Y_{1}}}{\partial{z_{11}}}\frac{\partial{z_{11}}}{\partial{x_1}} +
	\frac{\partial{Y_{2}}}{\partial{z_{11}}}\frac{\partial{z_{11}}}{\partial{x_1}} 
\right) +
\left(
	\frac{\partial{Y_{1}}}{\partial{z_{12}}}\frac{\partial{z_{12}}}{\partial{x_1}} +
	\frac{\partial{Y_{2}}}{\partial{z_{12}}}\frac{\partial{z_{12}}}{\partial{x_1}} 
\right) + \nonumber \\
&&
\left(
	\frac{\partial{Y_{3}}}{\partial{z_{21}}}\frac{\partial{z_{21}}}{\partial{x_1}} +
	\frac{\partial{Y_{4}}}{\partial{z_{21}}}\frac{\partial{z_{21}}}{\partial{x_1}} 
\right) +
\left(
	\frac{\partial{Y_{3}}}{\partial{z_{22}}}\frac{\partial{z_{22}}}{\partial{x_1}} +
	\frac{\partial{Y_{4}}}{\partial{z_{22}}}\frac{\partial{z_{22}}}{\partial{x_1}} 
\right) + \nonumber \\
&&
\left(
	\frac{\partial{Y_{5}}}{\partial{z_{31}}}\frac{\partial{z_{31}}}{\partial{x_1}} +
	\frac{\partial{Y_{6}}}{\partial{z_{31}}}\frac{\partial{z_{31}}}{\partial{x_1}} 
\right) +
\left(
	\frac{\partial{Y_{5}}}{\partial{z_{32}}}\frac{\partial{z_{32}}}{\partial{x_1}} +
	\frac{\partial{Y_{6}}}{\partial{z_{32}}}\frac{\partial{z_{32}}}{\partial{x_1}}
\right) \nonumber \\
& = &
\frac{\partial{E}}{\partial{z_{11}}}\frac{\partial{z_{11}}}{\partial{x_1}} +
\frac{\partial{E}}{\partial{z_{12}}}\frac{\partial{z_{12}}}{\partial{x_1}} +
\frac{\partial{E}}{\partial{z_{21}}}\frac{\partial{z_{21}}}{\partial{x_1}} +
\frac{\partial{E}}{\partial{z_{22}}}\frac{\partial{z_{22}}}{\partial{x_1}} +
\frac{\partial{E}}{\partial{z_{31}}}\frac{\partial{z_{31}}}{\partial{x_1}} +
\frac{\partial{E}}{\partial{z_{32}}}\frac{\partial{z_{32}}}{\partial{x_1}} \nonumber \\
& = &
\left[\begin{array}{cccccc}
\partial{E} / \partial{z_{11}} & \partial{E} / \partial{z_{12}} &
\partial{E} / \partial{z_{21}} & \partial{E} / \partial{z_{22}} &
\partial{E} / \partial{z_{31}} & \partial{E} / \partial{z_{32}}
\end{array}
\right]^T
\left[\begin{array}{c}
\partial{z_{11}} / \partial{x_1} \\
\partial{z_{12}} / \partial{x_1} \\
\partial{z_{21}} / \partial{x_1} \\
\partial{z_{22}} / \partial{x_1} \\
\partial{z_{31}} / \partial{x_1} \\
\partial{z_{32}} / \partial{x_1} \\
\end{array}
\right]
\end{eqnarray}

If the element-wise matrix multiplication 
(\href{https://en.wikipedia.org/wiki/Hadamard_product_(matrices)}{Hadamard product}) is 
denoted as $\circ$:

\begin{equation}
(A \circ B)_{i,j} = (A)_{i,j}(B)_{i,j}
\end{equation}

\noindent for any two matrices A and B of the same dimension and \textbf{grandsum} is the
sum of all elements of arbitrary matrix A of shape $[m, n]$: 

\begin{equation}
	\mathrm{grandsum}(A) = \sum_{i}^{m}{\sum_{j}^{n}{(A)_{i, j}}}
\end{equation}

\noindent Then equation 52 can be converted to the following form:

\begin{eqnarray}
\frac{\partial{E}}{\partial{x_1}} 
& = & 
\mathrm{grandsum}\left(
	\left[\begin{array}{ccc}
		\partial{E} / \partial{z_{11}} & 
		\partial{E} / \partial{z_{12}} &
		\partial{E} / \partial{z_{13}} \\ 
		\partial{E} / \partial{z_{21}} &
		\partial{E} / \partial{z_{22}} &
		\partial{E} / \partial{z_{23}} \\
		\partial{E} / \partial{z_{31}} &
		\partial{E} / \partial{z_{32}} &
		\partial{E} / \partial{z_{33}} \\
		\partial{E} / \partial{z_{41}} &
		\partial{E} / \partial{z_{42}} &
		\partial{E} / \partial{z_{43}} \\
		\end{array}
	\right] 
	\circ 
	\left[\begin{array}{ccc}
		\partial{z_{11}} / \partial{x_1} & 
		\partial{z_{12}} / \partial{x_1} &
		0 \\ 
		\partial{z_{21}} / \partial{x_1} &
		\partial{z_{22}} / \partial{x_1} &
		0 \\
		\partial{z_{31}} / \partial{x_1} &
		\partial{z_{32}} / \partial{x_1} &
		0 \\
		0 &
		0 &
		0 \\	
		\end{array}
	\right]
\right) \label{dE_dx1_grandsum_explicit} \\
& = &
\mathrm{grandsum}\left(
	\partial{E} / \partial{\mathbf{Z^{(0)}_{\mathrm{BBB}}}} 
	\circ
	\partial{\mathbf{Z}^{(0)}_{\mathbf{BBB}}} / \partial{x_1}
\right)
\end{eqnarray}

\noindent TensorFlow can handle the complicated derivative 
$\partial{E} / \partial{\mathbf{Z^{(0)}_{\mathrm{BBB}}}}$ and 
$\partial{\mathbf{Z}^{(0)}_{\mathrm{BBB}}} / \partial{x_1}$ can be pre-computed because it 
doesn't depend on kernel weights. 
However, $\partial{\mathbf{Z}^{(0)}_{\mathrm{BBB}}} / \partial{x_1}$ in equation 
\ref{dE_dx1_grandsum_explicit} has 12 ($C^4_3 \cdot C^3_2$) entries 
but only 6 ($C^4_3 \cdot C^3_2 \cdot 6 / (4 \cdot 3)$) of them are non-zero. 
To avoid unnecessary space waste, the coefficients matrix 
$\partial{\mathbf{Z}^{(0)}_{\mathrm{BBB}}} / \partial{\{x, y, z\}_i}$ should be constructed in
another way.

Since each entry of $\partial{E} / \partial{\mathbf{Z}^{(0)}_{\mathrm{BBB}}}$ contributes 
to six force components, it's natural for us to tile 
$\partial{E} / \partial{\mathbf{Z}^{(0)}_{\mathrm{BBB}}}$ six times so that each entry of the 
tiled matrix only contributes to one force component:

\begin{equation}
\left(\frac{\partial{E}}{\partial{\mathbf{Z}^{(0)}_{\mathrm{BBB}}}}\right)_{\mathrm{tiled}} = 
\left[\begin{array}{cccccc}
\partial{E} / \partial{\mathbf{Z}^{(0)}_{\mathrm{BBB}}} & 
\partial{E} / \partial{\mathbf{Z}^{(0)}_{\mathrm{BBB}}} &
\partial{E} / \partial{\mathbf{Z}^{(0)}_{\mathrm{BBB}}} & 
\partial{E} / \partial{\mathbf{Z}^{(0)}_{\mathrm{BBB}}} &
\partial{E} / \partial{\mathbf{Z}^{(0)}_{\mathrm{BBB}}} & 
\partial{E} / \partial{\mathbf{Z}^{(0)}_{\mathrm{BBB}}} 
\end{array}
\right]
\end{equation}

\noindent and then calculate the coefficients matrix 
$\partial{\mathbf{Z}^{(0)}_{\mathrm{BBB}}} / \partial{\{x, y, z\}_i}$:

\begin{eqnarray}
\left(\frac{
	\partial{\mathbf{Z}^{(0)}_{\mathrm{BBB}}}}{\partial{\{x, y, z\}_i}}
\right)^\mathbf{T}
& = &
\left[\begin{array}{cccc}
\partial{z_{11}} / \partial{x_1} & \partial{z_{21}} / \partial{x_1} &
\partial{z_{31}} / \partial{x_1} & \partial{z_{41}} / \partial{x_2} \\
\partial{z_{12}} / \partial{x_1} & \partial{z_{22}} / \partial{x_1} &
\partial{z_{32}} / \partial{x_1} & \partial{z_{42}} / \partial{x_2} \\
\partial{z_{13}} / \partial{x_2} & \partial{z_{23}} / \partial{x_2} &
\partial{z_{33}} / \partial{x_3} & \partial{z_{43}} / \partial{x_3} \\
\partial{z_{11}} / \partial{y_1} & \partial{z_{21}} / \partial{y_1} &
\partial{z_{31}} / \partial{y_1} & \partial{z_{41}} / \partial{y_2} \\
\partial{z_{12}} / \partial{y_1} & \partial{z_{22}} / \partial{y_1} &
\partial{z_{32}} / \partial{y_1} & \partial{z_{42}} / \partial{y_2} \\
\partial{z_{13}} / \partial{y_2} & \partial{z_{23}} / \partial{y_2} &
\partial{z_{33}} / \partial{y_3} & \partial{z_{43}} / \partial{y_3} \\
\partial{z_{11}} / \partial{z_1} & \partial{z_{21}} / \partial{z_1} &
\partial{z_{31}} / \partial{z_1} & \partial{z_{41}} / \partial{z_2} \\
\partial{z_{12}} / \partial{z_1} & \partial{z_{22}} / \partial{z_1} &
\partial{z_{32}} / \partial{z_1} & \partial{z_{42}} / \partial{z_2} \\
\partial{z_{13}} / \partial{z_2} & \partial{z_{23}} / \partial{z_2} &
\partial{z_{33}} / \partial{z_3} & \partial{z_{43}} / \partial{z_3} \\
\partial{z_{11}} / \partial{x_2} & \partial{z_{21}} / \partial{x_2} &
\partial{z_{31}} / \partial{x_3} & \partial{z_{41}} / \partial{x_3} \\
\partial{z_{12}} / \partial{x_3} & \partial{z_{22}} / \partial{x_4} &
\partial{z_{32}} / \partial{x_4} & \partial{z_{42}} / \partial{x_4} \\
\partial{z_{13}} / \partial{x_3} & \partial{z_{23}} / \partial{x_4} &
\partial{z_{33}} / \partial{x_4} & \partial{z_{43}} / \partial{x_4} \\
\partial{z_{11}} / \partial{y_2} & \partial{z_{21}} / \partial{y_2} &
\partial{z_{31}} / \partial{y_3} & \partial{z_{41}} / \partial{y_3} \\
\partial{z_{12}} / \partial{y_3} & \partial{z_{22}} / \partial{y_4} &
\partial{z_{32}} / \partial{y_4} & \partial{z_{42}} / \partial{y_4} \\
\partial{z_{13}} / \partial{y_3} & \partial{z_{23}} / \partial{y_4} &
\partial{z_{33}} / \partial{y_4} & \partial{z_{43}} / \partial{y_4} \\
\partial{z_{11}} / \partial{z_2} & \partial{z_{21}} / \partial{z_2} &
\partial{z_{31}} / \partial{z_3} & \partial{z_{41}} / \partial{z_3} \\
\partial{z_{12}} / \partial{z_3} & \partial{z_{22}} / \partial{z_4} &
\partial{z_{32}} / \partial{z_4} & \partial{z_{42}} / \partial{z_4} \\
\partial{z_{13}} / \partial{z_3} & \partial{z_{23}} / \partial{z_4} &
\partial{z_{33}} / \partial{z_4} & \partial{z_{43}} / \partial{z_4} \\
\end{array}
\right]
\end{eqnarray}

\noindent Then we can obtain the force contributions matrix,
$\partial{E} / \partial{\{x, y, z\}_i}$, with the following Hadamard product: 

\begin{equation}
\frac{\partial{E}}{\partial{\{x, y, z\}_i}}
=
\frac{\partial{\mathbf{Z}^{(0)}_{\mathrm{BBB}}}}{\partial{\{x, y, z\}_i}}
\circ
\left(
	\frac{\partial{E}}{\partial{\mathbf{Z}^{(0)}_{\mathrm{BBB}}}}
\right)_{\mathrm{tiled}}
\end{equation}

\noindent Expand $\partial{E} / \partial{\{x, y, z\}_i}$ we can have:

\begin{equation}
\frac{\partial{E}}{\partial{\{x, y, z\}_i}} = \left[\begin{array}{cccc}
\partial{E} / \partial{z_{11}} \cdot \partial{z_{11}} / \partial{x_1} & 
\partial{E} / \partial{z_{21}} \cdot \partial{z_{21}} / \partial{x_1} &
\partial{E} / \partial{z_{31}} \cdot \partial{z_{31}} / \partial{x_1} & 
\partial{E} / \partial{z_{41}} \cdot \partial{z_{41}} / \partial{x_2} \\
\partial{E} / \partial{z_{12}} \cdot \partial{z_{12}} / \partial{x_1} & 
\partial{E} / \partial{z_{22}} \cdot \partial{z_{22}} / \partial{x_1} &
\partial{E} / \partial{z_{32}} \cdot \partial{z_{32}} / \partial{x_1} & 
\partial{E} / \partial{z_{42}} \cdot \partial{z_{42}} / \partial{x_2} \\
\partial{E} / \partial{z_{13}} \cdot \partial{z_{13}} / \partial{x_2} & 
\partial{E} / \partial{z_{23}} \cdot \partial{z_{23}} / \partial{x_2} &
\partial{E} / \partial{z_{33}} \cdot \partial{z_{33}} / \partial{x_3} & 
\partial{E} / \partial{z_{43}} \cdot \partial{z_{43}} / \partial{x_3} \\
\partial{E} / \partial{z_{11}} \cdot \partial{z_{11}} / \partial{y_1} & 
\partial{E} / \partial{z_{21}} \cdot \partial{z_{21}} / \partial{y_1} &
\partial{E} / \partial{z_{31}} \cdot \partial{z_{31}} / \partial{y_1} & 
\partial{E} / \partial{z_{41}} \cdot \partial{z_{41}} / \partial{y_2} \\
\partial{E} / \partial{z_{12}} \cdot \partial{z_{12}} / \partial{y_1} & 
\partial{E} / \partial{z_{22}} \cdot \partial{z_{22}} / \partial{y_1} &
\partial{E} / \partial{z_{32}} \cdot \partial{z_{32}} / \partial{y_1} & 
\partial{E} / \partial{z_{42}} \cdot \partial{z_{42}} / \partial{y_2} \\
\partial{E} / \partial{z_{13}} \cdot \partial{z_{13}} / \partial{y_2} & 
\partial{E} / \partial{z_{23}} \cdot \partial{z_{23}} / \partial{y_2} &
\partial{E} / \partial{z_{33}} \cdot \partial{z_{33}} / \partial{y_3} & 
\partial{E} / \partial{z_{43}} \cdot \partial{z_{43}} / \partial{y_3} \\
\partial{E} / \partial{z_{11}} \cdot \partial{z_{11}} / \partial{z_1} & 
\partial{E} / \partial{z_{21}} \cdot \partial{z_{21}} / \partial{z_1} &
\partial{E} / \partial{z_{31}} \cdot \partial{z_{31}} / \partial{z_1} & 
\partial{E} / \partial{z_{41}} \cdot \partial{z_{41}} / \partial{z_2} \\
\partial{E} / \partial{z_{12}} \cdot \partial{z_{12}} / \partial{z_1} & 
\partial{E} / \partial{z_{22}} \cdot \partial{z_{22}} / \partial{z_1} &
\partial{E} / \partial{z_{32}} \cdot \partial{z_{32}} / \partial{z_1} & 
\partial{E} / \partial{z_{42}} \cdot \partial{z_{42}} / \partial{z_2} \\
\partial{E} / \partial{z_{13}} \cdot \partial{z_{13}} / \partial{z_2} & 
\partial{E} / \partial{z_{23}} \cdot \partial{z_{23}} / \partial{z_2} &
\partial{E} / \partial{z_{33}} \cdot \partial{z_{33}} / \partial{z_3} & 
\partial{E} / \partial{z_{43}} \cdot \partial{z_{43}} / \partial{z_3} \\
\partial{E} / \partial{z_{11}} \cdot \partial{z_{11}} / \partial{x_2} & 
\partial{E} / \partial{z_{21}} \cdot \partial{z_{21}} / \partial{x_2} &
\partial{E} / \partial{z_{31}} \cdot \partial{z_{31}} / \partial{x_3} & 
\partial{E} / \partial{z_{41}} \cdot \partial{z_{41}} / \partial{x_3} \\
\partial{E} / \partial{z_{12}} \cdot \partial{z_{12}} / \partial{x_3} & 
\partial{E} / \partial{z_{22}} \cdot \partial{z_{22}} / \partial{x_4} &
\partial{E} / \partial{z_{32}} \cdot \partial{z_{32}} / \partial{x_4} & 
\partial{E} / \partial{z_{42}} \cdot \partial{z_{42}} / \partial{x_4} \\
\partial{E} / \partial{z_{13}} \cdot \partial{z_{13}} / \partial{x_3} & 
\partial{E} / \partial{z_{23}} \cdot \partial{z_{23}} / \partial{x_4} &
\partial{E} / \partial{z_{33}} \cdot \partial{z_{33}} / \partial{x_4} & 
\partial{E} / \partial{z_{43}} \cdot \partial{z_{43}} / \partial{x_4} \\
\partial{E} / \partial{z_{11}} \cdot \partial{z_{11}} / \partial{y_2} & 
\partial{E} / \partial{z_{21}} \cdot \partial{z_{21}} / \partial{y_2} &
\partial{E} / \partial{z_{31}} \cdot \partial{z_{31}} / \partial{y_3} & 
\partial{E} / \partial{z_{41}} \cdot \partial{z_{41}} / \partial{y_3} \\
\partial{E} / \partial{z_{12}} \cdot \partial{z_{12}} / \partial{y_3} & 
\partial{E} / \partial{z_{22}} \cdot \partial{z_{22}} / \partial{y_4} &
\partial{E} / \partial{z_{32}} \cdot \partial{z_{32}} / \partial{y_4} & 
\partial{E} / \partial{z_{42}} \cdot \partial{z_{42}} / \partial{y_4} \\
\partial{E} / \partial{z_{13}} \cdot \partial{z_{13}} / \partial{y_3} & 
\partial{E} / \partial{z_{23}} \cdot \partial{z_{23}} / \partial{y_4} &
\partial{E} / \partial{z_{33}} \cdot \partial{z_{33}} / \partial{y_4} & 
\partial{E} / \partial{z_{43}} \cdot \partial{z_{43}} / \partial{y_4} \\
\partial{E} / \partial{z_{11}} \cdot \partial{z_{11}} / \partial{z_2} & 
\partial{E} / \partial{z_{21}} \cdot \partial{z_{21}} / \partial{z_2} &
\partial{E} / \partial{z_{31}} \cdot \partial{z_{31}} / \partial{z_3} & 
\partial{E} / \partial{z_{41}} \cdot \partial{z_{41}} / \partial{z_3} \\
\partial{E} / \partial{z_{12}} \cdot \partial{z_{12}} / \partial{z_3} & 
\partial{E} / \partial{z_{22}} \cdot \partial{z_{22}} / \partial{z_4} &
\partial{E} / \partial{z_{32}} \cdot \partial{z_{32}} / \partial{z_4} & 
\partial{E} / \partial{z_{42}} \cdot \partial{z_{42}} / \partial{z_4} \\
\partial{E} / \partial{z_{13}} \cdot \partial{z_{13}} / \partial{z_3} & 
\partial{E} / \partial{z_{23}} \cdot \partial{z_{23}} / \partial{z_4} &
\partial{E} / \partial{z_{33}} \cdot \partial{z_{33}} / \partial{z_4} & 
\partial{E} / \partial{z_{43}} \cdot \partial{z_{43}} / \partial{z_4} \\
\end{array}
\right]
\end{equation}

\noindent Now, the one last thing to do is building an auxiliary matrix $\mathbf{IND}$ that can 
\textbf{re-order} the entries of $\partial{E} / \partial{\{x, y, z\}_i}$ 
to build a matrix of shape $[3\mathrm{N}, 6]$ so that all entries of each row corresponds to 
the same force component. \textbf{IND} can also be pre-computed.

\begin{eqnarray}
&& \left(
	\frac{\partial{E}}{\partial{\{x, y, z\}_i}}
\right)_{\mathrm{ordered}} \nonumber \\
& = &
\mathbf{Reorder}\left(
	\frac{\partial{E}}{\partial{\{x, y, z\}_i}}, \mathbf{IND}
\right) \nonumber \\
& = & \begin{blockarray}{ccccccc}
Component & & & & & & \\
\begin{block}{c(cccccc)}
f^x_1 &
\partial{E^{11}} / \partial{x_1} & \partial{E^{12}} / \partial{x_1} &
\partial{E^{21}} / \partial{x_1} & \partial{E^{22}} / \partial{x_1} &
\partial{E^{31}} / \partial{x_1} & \partial{E^{32}} / \partial{x_1} \\
f^y_1 &
\partial{E^{11}} / \partial{y_1} & \partial{E^{12}} / \partial{y_1} &
\partial{E^{21}} / \partial{y_1} & \partial{E^{22}} / \partial{y_1} &
\partial{E^{31}} / \partial{y_1} & \partial{E^{32}} / \partial{y_1} \\
f^z_1 &
\partial{E^{11}} / \partial{z_1} & \partial{E^{12}} / \partial{z_1} &
\partial{E^{21}} / \partial{z_1} & \partial{E^{22}} / \partial{z_1} &
\partial{E^{31}} / \partial{z_1} & \partial{E^{32}} / \partial{z_1} \\
%
f^x_2 &
\partial{E^{13}} / \partial{x_2} & \partial{E^{11}} / \partial{x_2} &
\partial{E^{23}} / \partial{x_2} & \partial{E^{21}} / \partial{x_2} &
\partial{E^{41}} / \partial{x_2} & \partial{E^{42}} / \partial{x_2} \\
f^y_2 &
\partial{E^{13}} / \partial{y_2} & \partial{E^{11}} / \partial{y_2} &
\partial{E^{23}} / \partial{y_2} & \partial{E^{21}} / \partial{y_2} &
\partial{E^{41}} / \partial{y_2} & \partial{E^{42}} / \partial{y_2} \\
f^z_2 &
\partial{E^{13}} / \partial{z_2} & \partial{E^{11}} / \partial{z_2} &
\partial{E^{23}} / \partial{z_2} & \partial{E^{21}} / \partial{z_2} &
\partial{E^{41}} / \partial{z_2} & \partial{E^{42}} / \partial{z_2} \\
%
f^x_3 &
\partial{E^{12}} / \partial{x_3} & \partial{E^{13}} / \partial{x_3} &
\partial{E^{33}} / \partial{x_3} & \partial{E^{31}} / \partial{x_3} &
\partial{E^{43}} / \partial{x_3} & \partial{E^{41}} / \partial{x_3} \\
f^y_3 &
\partial{E^{12}} / \partial{y_3} & \partial{E^{13}} / \partial{y_3} &
\partial{E^{33}} / \partial{y_3} & \partial{E^{31}} / \partial{y_3} &
\partial{E^{43}} / \partial{y_3} & \partial{E^{41}} / \partial{y_3} \\
f^z_3 &
\partial{E^{12}} / \partial{z_3} & \partial{E^{13}} / \partial{z_3} &
\partial{E^{33}} / \partial{z_3} & \partial{E^{31}} / \partial{z_3} &
\partial{E^{43}} / \partial{z_3} & \partial{E^{41}} / \partial{z_3} \\
%
f^x_4 &
\partial{E^{22}} / \partial{x_4} & \partial{E^{23}} / \partial{x_4} &
\partial{E^{32}} / \partial{x_4} & \partial{E^{33}} / \partial{x_4} &
\partial{E^{42}} / \partial{x_4} & \partial{E^{43}} / \partial{x_4} \\
f^y_4 &
\partial{E^{22}} / \partial{y_4} & \partial{E^{23}} / \partial{y_4} &
\partial{E^{32}} / \partial{y_4} & \partial{E^{33}} / \partial{y_4} &
\partial{E^{42}} / \partial{y_4} & \partial{E^{43}} / \partial{y_4} \\
f^z_4 &
\partial{E^{22}} / \partial{z_4} & \partial{E^{23}} / \partial{z_4} &
\partial{E^{32}} / \partial{z_4} & \partial{E^{33}} / \partial{z_4} &
\partial{E^{42}} / \partial{z_4} & \partial{E^{43}} / \partial{z_4} \\
\end{block}
\end{blockarray}
\end{eqnarray}

\noindent where:

\begin{equation}
\frac{\partial{E^{ab}}}{\partial{\{x, y, z\}_i}} = 
\frac{\partial{E}}{\partial{z_{ab}}}
\cdot
\frac{\partial{z_{ab}}}{\partial{\{x, y, z\}_3}}
\end{equation}

\noindent Finally, we can get all atomic forces by summing up each row:

\begin{equation}
\vec{F}(\mathrm{B}_{4}) = 
\mathbf{Sum}\left(
\left(
	\frac{\partial{E}}{\partial{\{x, y, z\}_i}}
\right)_{\mathrm{ordered}}, \quad \mathrm{axis} = 1
\right)
\end{equation}

\end{document}
