\documentclass{article}

\usepackage{amsmath}
\usepackage[margin=1in]{geometry}
\usepackage{graphicx}

\title{The theory of KCNN}

\author{Xin Chen}
\date{\today}


\begin{document}
\maketitle
	

\section{Energy}

The energy of KCNN is modeled under the many-body expression scheme:

\begin{eqnarray}
E^{total} = \sum_{a}^{C^N_1}{E^{(k=1)}_{a}} + \sum_{a,b}^{C^{N}_2}{E^{(k=2)}_{ab}} 
+ \sum_{a,b,c}^{C^{N}_3}{E^{(k=3)}_{abc}} + \cdots
\end{eqnarray}

\noindent For most cases, the equation above can be truncated at $k = 3$. Higher order 
terms contribute far less to the total energy while require much more computational
resources as $C^N_k$ roughly scales as $\mathcal{O}(N)$.

\subsection{One-body}

In KCNN, the 1-body terms are expressed by a linear model:

\begin{eqnarray}
\sum_{a}^{C^N_1}{E^{(k=1)}_{a}} = \sum_{a}^{N}{E^{A_a}} = \sum_{A_{a}}{n^{A_a}E^{A_a}}
\end{eqnarray}

\noindent As an example, the 1-body contribution of a 
$\mathrm{C}_9 \mathrm{H}_7 \mathrm{N}$ is:

\begin{eqnarray}
E^{(k=1)} = 9E^{\mathrm{C}} + 7E^{\mathrm{H}} + E^{\mathrm{N}}
\end{eqnarray}

\noindent Here $E^{\mathrm{C}}$, $E^{\mathrm{H}}$ and $E^{\mathrm{N}}$ are all
trainable parameters.

\subsection{Two/three-body}

Independent convolutional neural networks are used to fit two/three-body terms:

\begin{eqnarray}
E^{(k=2)} & = & \sum_{a,b}^{C^N_2}{
	\boldmath{\mathrm{CNN}}^{\mathrm{A}_{a}\mathrm{A}_{b}}
}(r_{ab}) \\
E^{(k=3)} & = & \sum_{a,b,c}^{C^N_3}{
	\boldmath{\mathrm{CNN}}^{\mathrm{A}_{a}\mathrm{A}_{b}\mathrm{A}_{c}}
}(r_{ab}, r_{ac}, r_{bc})
\end{eqnarray}

To simplify the equations and normalize the distances $r$, an exponential 
transformation is adopted:

\begin{eqnarray}
z_{ab} = \exp{\left(-\frac{r_{ab}}{L_{A_a} + L_{A_b}}\right)}
\end{eqnarray}

\noindent where $L_{A_a}$ is the covalent radius of element $A_{a}$. For non periodic 
molecules the Pÿÿkko radii are used.

\subsection{The construction of the input feature matrix}

Now we can build the input feature matrix in a straightforward way. Taking the example 
of $\mathrm{C}_9 \mathrm{H}_7 \mathrm{N}$, the terms and their dimensions are:

\begin{center}
	\scalebox{1.0}{
		\begin{tabular}{l*{3}{c}r}
			Term              & k & Expression & Dimension \\
			\hline
			CC    & 2 & $C^9_2$                         &  36 \\
			CH    & 2 & $C^9_1 \cdot C^7_1$             &  63 \\
			CN    & 2 & $C^9_1 \cdot C^1_1$             &   9 \\
			HH    & 2 & $C^7_2$                         &  21 \\
			HN    & 2 & $C^7_1 \cdot C^1_1$             &   7 \\
			CCC   & 3 & $C^9_3$                         &  84 \\
			CCH   & 3 & $C^9_2 \cdot C^7_1$             & 252 \\
			CCN   & 3 & $C^9_2$                         &  36 \\
			CHH   & 3 & $C^9_1 \cdot C^7_2$             & 189 \\
			CHN   & 3 & $C^9_1 \cdot C^7_1 \cdot C^1_1$ &  63 \\
			HHH   & 3 & $C^7_3$                         &  35 \\
			HHN   & 3 & $C^7_2 \cdot C^1_1$             &  21 \\
			Total &   &                                 & 816 
		\end{tabular}	
	}
\end{center}

One can notice that $\mathrm{C}_9 \mathrm{H}_7 \mathrm{N}$ has 17 atoms but $C^{18}_3=816$.
The CC and CCN, CH and CHN, HH and HHN all have the same dimension because $\mathrm{C}_9 
\mathrm{H}_7 \mathrm{N}$ only has one nitrogen atom. So here we propose a 'ghost atom' 
scheme which can simplify the construction of the input feature matrix. 
A ghost atom, denoted as X, is appended. Its distances to any real atom is $+\infty$. 
According to equation 6, the scaled distance of X to any real atom will be zero. 
Then we drop all previous 2-body terms and only keep 3-body terms:

\begin{center}
	\scalebox{1.0}{
		\begin{tabular}{l*{3}{c}r}
			Term              & k & Expression & Dimension \\
			\hline
			CCX   & 3 & $C^9_2$                         &  36 \\
			CHX   & 3 & $C^9_1 \cdot C^7_1$             &  63 \\
			CNX   & 3 & $C^9_1 \cdot C^1_1$             &   9 \\
			HHX   & 3 & $C^7_2$                         &  21 \\
			HNX   & 3 & $C^7_1 \cdot C^1_1$             &   7 \\
			CCC   & 3 & $C^9_3$                         &  84 \\
			CCH   & 3 & $C^9_2 \cdot C^7_1$             & 252 \\
			CCN   & 3 & $C^9_2$                         &  36 \\
			CHH   & 3 & $C^9_1 \cdot C^7_2$             & 189 \\
			CHN   & 3 & $C^9_1 \cdot C^7_1 \cdot C^1_1$ &  63 \\
			HHH   & 3 & $C^7_3$                         &  35 \\
			HHN   & 3 & $C^7_2 \cdot C^1_1$             &  21 \\
			Total &   &                                 & 816 
		\end{tabular}	
	}
\end{center}

Now we only have 3-body terms but the dimensions are unchanged. With this scheme, we can 
use just one matrix to store all 2-body and 3-body features.

\subsection{Permutational Invariance}

The three spatial (translational, rotational and permutational) invariances must all be 
satisfied. Since KCNN only uses interatomic distances $r$, the translational and 
rotational invariances are naturally kept and the 1-body and 2-body terms are also 
permutationally invariant. However, the 3-body terms \(or higher\) do not satisfy this
requirement because the orders of the parameters of convolutional kernels are fixed 
while $(\mathrm{A}_{a}\mathrm{A}_{b}\mathrm{A}_{c})$ are inter-changeable. 

To overcome this problem, the conditional sorting scheme is adopted: the columns of the 
input matrix (the input layer) of each k-body CNN are ordered according to the bond types, 
and for each k-body interaction (matrix row) we only sort the  entries of the same atom 
types. Taking the examples of CHN, CCH and CCC:

\begin{itemize}
	\item CHN: each column represents a unique bond type (C-H, C-N, H-N). Sorting is not needed.
	\item CCN: the entries corresponding to C-C of each row should be sorted.
	\item CCC: all three entries of each row should be sorted.
\end{itemize}


\section{Force}

Atomic force is the negative first-order derivative of energy with respect to 
displacement:

\begin{eqnarray}
f(r) = -\frac{\partial E(r)}{\partial r}
\end{eqnarray}

\noindent So it's straightforward to get KCNN atomic forces:

\begin{eqnarray}
f(x_{i}) & = & -\frac{\partial{E^{total}}}{\partial{x_{i}}} \nonumber \\
& = & -\left(
	\frac{\partial{E^{(k=1)}}}{\partial{x_i}} 
	+ \frac{\partial{E^{(k=2)}}}{\partial{x_i}}
	+ \frac{\partial{E^{(k=3)}}}{\partial{x_i}} 
\right) \nonumber \\
& = & -\left(
\frac{
	\partial{\sum_{a,b}^{C^N_2}{
		\boldmath{\mathrm{CNN}}^{\mathrm{A}_{a}\mathrm{A}_{b}}}(z_{ab})}
	}{
		\partial{x_i}
	} 
+ 
\frac{
	\partial{\sum_{a,b,c}^{C^N_3}{
		\boldmath{\mathrm{CNN}}^{\mathrm{A}_{a}\mathrm{A}_{b}\mathrm{A}_{c}}}
		(z_{ab}, z_{ac}, z_{bc})}
	}{
		\partial{x_i}
	} 
\right) \nonumber \\
& = & -\left(
\sum_{a,b}^{C^N_2}{
	\frac{
		\partial{\boldmath{\mathrm{CNN}}^{\mathrm{A}_{a}\mathrm{A}_{b}}(z_{ab})}
	}{
		\partial{x_i}
	}
}
	+
\sum_{a,b,c}^{C^N_3}{
	\frac{
		\partial{\boldmath{\mathrm{CNN}}^{\mathrm{A}_{a}\mathrm{A}_{b}}
		(z_{ab}, z_{ac}, z_{bc})}
	}{
		\partial{x_i}
	}
}
\right)
\end{eqnarray}

\noindent where $f(x_{i})$ is the force component of atom $i$ along the X direction. We 
also have:

\begin{eqnarray}
\frac{
	\partial{\boldmath{\mathrm{CNN}}^{\mathrm{A}_{a}\mathrm{A}_{b}}(z_{ab})}
}{
	\partial{x_i}
} 
& = & 
\frac{\partial{
	\boldmath{\mathrm{CNN}}^{\mathrm{A}_{a}\mathrm{A}_{b}}(z_{ab})}
}{
	\partial{z_{ab}}
} \frac{\partial{z_{ab}}}{\partial{r_{ab}}} \frac{\partial{r_{ab}}}{\partial{x_i}} \\
\frac{ \partial{z_{ab}}}{\partial{r_{ab}}} 
& = & 
-\frac{z_{ab}}{L_{A_a} + L_{A_b}} = -\frac{z_{ab}}{L_{ab}} \\
\frac{\partial{r_{ab}}}{\partial{x_i}} & = & \begin{cases}
	\frac{x_i - x_b}{r_{ib}} & \quad \text{if } i = a \\
	\frac{x_a - x_i}{r_{ai}} & \quad \text{if } i = b \\
	0                        & \quad \text{else}
\end{cases} 
\end{eqnarray}

\end{document}
